%%---Main Packages----------------------------------------------------------------------

\usepackage[english, ngerman]{babel}	%Mul­tilin­gual sup­port for Plain TeX or LaTeX
\usepackage[T1]{fontenc}				%Stan­dard pack­age for se­lect­ing font en­codings
\usepackage[utf8]{inputenc}				%Ac­cept dif­fer­ent in­put en­cod­ings
\usepackage{graphicx} 					%En­hanced sup­port for graph­ics
\usepackage{float}						%Im­proved in­ter­face for float­ing ob­jects


%%---Optional Packages------------------------------------------------------------------

\usepackage[pdftex,dvipsnames,table]{xcolor}%Driver-in­de­pen­dent color ex­ten­sions for LaTeX
\usepackage[hidelinks]{hyperref}		%Ex­ten­sive sup­port for hy­per­text in LaTeX
\usepackage{amsmath}					%AMS math­e­mat­i­cal fa­cil­i­ties for LaTeX
%\usepackage{amsthm}					%Type­set­ting the­o­rems (AMS style)
%\usepackage{amssymb}					%Type­set­ting the­o­rems (AMS style)
\usepackage{pdfpages}					%In­clude PDF doc­u­ments in LaTeXmns
\usepackage{todonotes}					%Mark­ing things to do in a LaTeX doc­u­ment
%\usepackage{siunitx} 					%A com­pre­hen­sive (SI) units pack­age
\usepackage{listings}					%Type­set source code list­ings us­ing LaTeX
%\usepackage{pdflscape}					%Make land­scape pages dis­play as land­scape
%\usepackage{array}						%Ex­tend­ing the ar­ray and tab­u­lar en­vi­ron­ments
%\usepackage{imakeidx}					%A pack­age for pro­duc­ing mul­ti­ple in­dexes
\usepackage{subfigure}					%Fig­ures di­vided into sub­fig­ures
%\usepackage{booktabs}					%Publi­ca­tion qual­ity ta­bles in LaTeX
%\usepackage{multirow}					%Create tab­u­lar cells span­ning mul­ti­ple rows
%\usepackage{multicol}					%In­ter­mix sin­gle and mul­ti­ple columns
\usepackage{tabularx}					%Tab­u­lars with ad­justable-width columns
%\usepackage{xargs}                     %De­fine com­mands with many op­tional ar­gu­ments
%\usepackage{cite}						%Im­proved ci­ta­tion han­dling in LaTeX
\usepackage{setspace}					%Set space be­tween lines
%\usepackage{textgreek}					%Upright greek let­ters in text
%\usepackage[bottom]{footmisc}			%A range of foot­note op­tions
%\usepackage{footnote}					%Im­prove on LaTeX's foot­note han­dling
%\usepackage{verbatim}					%Reim­ple­men­ta­tion of and ex­ten­sions to LaTeX ver­ba­tim
\usepackage{textcomp}					%LaTeX sup­port for the Text Com­pan­ion fonts
%\usepackage[pagesize]{typearea}		%Set page mar­gins
%\usepackage{xspace}
%\usepackage{courier}                   %Monospaced font for listings package


%%---My Packages------------------------------------------------------------------------

\usepackage[left=2cm, right=2cm, bottom=2cm, headsep=1.5cm]{geometry} % Einstellung Seitenränder
\usepackage{lmodern}
\usepackage{scrpage2}					%Kopf- und Fusszeile
\usepackage{lscape}
\usepackage{tikz}						%Organigramm
\usetikzlibrary{trees}
\usetikzlibrary{mindmap}
\usetikzlibrary{positioning}
\usepackage{abstract}
\usepackage{etoolbox}


%%---Definitions----------------- ------------------------------------------------------

%%Tabel-Definitions: (requires \usepackage{tabularx})
%\newcolumntype{L}[1]{>{\raggedright\arraybackslash}p{#1}}    %column-width and alignment
%\newcolumntype{C}[1]{>{\centering\arraybackslash}p{#1}}
%\newcolumntype{R}[1]{>{\raggedleft\arraybackslash}p{#1}}


%%---Settings---------------------------------------------------------------------------

\bibliographystyle{IEEEtran}			%Defines the Bib. Style (Harvard, IEEE etc.)
\graphicspath{{./images/}}				%Defines the graphicspath
%\geometry{twoside=false}				%twoside=false disables the "bookstyle" => one sided document
\setlength{\marginparwidth}{2cm}
\overfullrule=5em						%Creates a black rule if text goes over the margins => debugging
\setlength{\parindent}{0pt}
\pagestyle{scrheadings}
\clearscrheadfoot
\ohead{\headmark}
\automark[section]{chapter}
\setheadsepline{0.4pt}

%%Listings-Settings: (requires \usepackage{listings})
% Example with Matlab Code:
%\lstset{language=Matlab,%
%    basicstyle=\footnotesize\ttfamily,
%    breaklines=false,%
%    morekeywords={matlab2tikz},
%    keywordstyle=\color{blue},%
%    tabsize=4,
%    morekeywords=[2]{1}, keywordstyle=[2]{\color{black}},
%    identifierstyle=\color{black},%
%    stringstyle=\color{mylilas},
%    commentstyle=\color{mygreen},%
%    showstringspaces=false,%without this there will be a symbol in the places where there is a space
%    numbers=left,%
%    numberstyle={\tiny \color{black}},% size of the numbers
%    numbersep=9pt, % this defines how far the numbers are from the text
%    emph=[1]{switch, case, otherwise, for,end,break},emphstyle=[1]\color{red}, %some words to emphasise
%    %emph=[2]{word1,word2}, emphstyle=[2]{style},    
%}

\definecolor{mGreen}{rgb}{0,0.6,0}
\definecolor{mGray}{rgb}{0.5,0.5,0.5}
\definecolor{mPurple}{rgb}{0.58,0,0.82}
\definecolor{backgroundColour}{rgb}{0.95,0.95,0.92}

\lstdefinestyle{CStyle}{
    backgroundcolor=\color{backgroundColour},   
    commentstyle=\color{mGreen},
    keywordstyle=\color{magenta},
    numberstyle=\tiny\color{mGray},
    stringstyle=\color{mPurple},
    basicstyle=\footnotesize\ttfamily,
    breakatwhitespace=false,         
    breaklines=true,                 
    captionpos=b,                    
    keepspaces=true,                 
    numbers=left,                    
    numbersep=5pt,                  
    showspaces=false,                
    showstringspaces=false,
    showtabs=false,                  
    tabsize=2,
    language=C
}
